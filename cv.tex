\documentclass{article}
\linespread{1}
\usepackage{geometry}
\geometry{
 left=2in,
 right=1in,
 top=1in,
 bottom=1in
 }
\usepackage[usenames,dvipsnames]{color}
\usepackage{fontspec} 
\usepackage{verse} 
\usepackage[square,sort,comma,numbers]{natbib} 
\usepackage{titlesec}
\usepackage{bibentry} 
\usepackage[urlcolor=Blue,colorlinks=true]{hyperref}
\usepackage[multiple]{footmisc}
\widowpenalty=10000
\clubpenalty=10000

\nobibliography* 

\newcommand{\stopstanza}{\\[\normalbaselineskip]}

\makeatletter
\def\subject#1{\gdef\@subject{#1}}
\def\memonumber#1{\gdef\@memonumber{#1}}
\def\@subject{\@latex@warning@no@line{No \noexpand\subject given}}
%... and your other categories as you wish,
% defining them by default to something for example
%
\renewcommand\maketitle{\par
  \begingroup
  \hskip -12em{\Huge \@author \par}
  \vskip 2em%
  \hskip -12em{\Large
%   \lineskip .75em%
    % \begin{tabular}[t]{c}%
      \@title
%    \end{tabular}
\par}%
    \vskip 6.5em%
%... the layout that you want, using \@author, \@subject, etc...
  \endgroup
}

\makeatother

\setromanfont{Hoefler Text}
\setsansfont{Optima}

\titleformat{\section}[leftmargin] 
{\bfseries} 
{}{0em}{} 
\titlespacing{\section} 
{4.5pc}{1.5ex plus .1ex minus .2ex}{3pc}


\titleformat{\subsection}
{\bfseries}{}{0em}{}

\setlength{\itemsep}{0pt}%
\setlength{\topsep}{0pt}%
\setlength{\parskip}{0pt}%

\title{Curriculum Vitae}
\author{\textcolor{BrickRed}{Fernando Diaz}}
\begin{document}
\maketitle

% \section{Research \\Interests}
% \noindent user interaction modeling, crisis informatics, evaluation, social media analysis, large-scale implicit user feedback, distributed information retrieval, user modeling, summarization, deep learning, fairness and transparency\\

\section{Education}\noindent\begin{tabbing}\setlength{\parskip}{0pt}\setlength{\topsep}{0pt}PhD~~~~\=\textbf{University of Massachusetts Amherst}\\
	\>Computer Science\` 2008\\
	\>“Regularizing Query-Based Retrieval Scores”\\
	\>	James Allan (chair), W. Bruce Croft, Sridhar Mahadevan, John Staudenmayer\\
\\
MS\>\textbf{University of Massachusetts Amherst}\\
	\>Computer Science\` 2004\\
	\>	“Browsing-Based User Language Models”\\
	\>	James Allan (advisor)\\
\\
BS\>\textbf{University of Michigan Ann Arbor}\\
	\>Computer Science\` 1998\\
BA\>\textbf{University of Michigan Ann Arbor}\\
	\>Political Science\` 1998\\
\end{tabbing}

\section{Research \\Experience}\noindent\textbf{Microsoft Research}\hfill Principal Research Manager\\
Montréal, QC\hfill June 2018-present\\\\
I lead a multidisciplinary research group focused on fairness, accountability, transparency, and ethics (FATE).  The group externally publishes basic research and actively collaborates with engineering teams in Microsoft.  In addition, I maintain an active research agenda in core information retrieval and its application to production systems such as bing. \\

\noindent\textbf{Spotify}\hfill Director of Research\\
New York, NY\hfill March 2017-May 2018\\\\
I led the applied research organization of twenty researchers focused on search, recommendation, metrics, and evaluation.  I helped define the company's research approach and culture, including how it interacts internally with product teams and externally with the academic community.  In addition, I was an active member of the recommendations leadership team, involved in strategic planning for a two year product roadmap.\\

\noindent\textbf{Microsoft Research}\hfill Senior Researcher\\
New York, NY\hfill May 2012-March 2017\\\\
I led and participated in information retrieval research projects covering crisis informatics,  attention modeling, fairness in machine learning, text summarization, and deep learning for search.  These themes included extensive collaboration internally with product groups and externally with academic institutions. \\
% \newpage

\noindent\textbf{Yahoo Research}\hfill Senior Research Scientist\\
New York, NY\hfill January 2008–March 2012\\\\
I led and participated in information retrieval research projects including federated search, time-sensitive web search, match-making systems, multi-document summarization, and modeling user mouse behavior.  My role balanced developing production code, technology transfer from research to applied research and engineering groups, and presentation of research results to the wider research community.  \\


\noindent\textbf{Institute for Pure and Applied Mathematics}\hfill Visiting Fellow\\
Los Angeles, CA\hfill September 2007–December 2007\\\\
I was a visiting fellow at UCLA for the program in ``Mathematics of Knowledge and Search Engines''.\\

\noindent\textbf{University of Massachusetts}\hfill Research Assistant\\
Amherst, MA\hfill  September 2001–August 2007\\\\
Disseration research focused on information retrieval under the supervision of James Allan.\\
% My dissertation research focused on black-box methods for improving retrieval systems. I designed an algorithm which significantly improves the performance of arbitrary retrieval systems. The generality of this algorithm meant that the result is significant for all areas of information retrieval. In addition to this empirical result, I drew theoretical connections to classic information retrieval techniques. Other results include the development of a robust measure for predicting the quality of an arbitrary document ranking and the design of an algorithm for topically aligning multilingual corpora without translation dictionaries.
%
% In collaboration with Professor James Allan, I have participated in other areas of information management including online summarization of news streams, statistical language modeling for personalization, and formal modeling of interactive information retrieval. Additionally, I assisted in the coordination of the High Accuracy Retrieval from Documents interactive search task for the National Institute of Standards and Technology (NIST) annual Text Retrieval Conference (TREC).\\


\noindent\textbf{Overture Research (now Yahoo Research) }\hfill Intern\\
Pasadena, CA\hfill  June 2003–December 2003\\\\
I collaborated with Dr. Rosie Jones to develop tools for the analysis and classification of search queries into temporal categories. \\%We studied the temporal properties of queries showing a correlation with retrieval performance. This result revealed that temporal features play a role in the task of performance prediction. In addition, I also worked with Dr. Daniel Fain on key phrase extraction from news articles. I implemented state of the art key phrase extraction technology and developed several statistical natural language processing tools.\\

\noindent\textbf{Sony Computer Science Laboratories }\hfill Intern\\
Tokyo, Japan\hfill  June 2001–August 2001\\\\
As part of participation in the NSF Summer Institute in Japan, I worked with Dr. Hitoshi Iida and Dr. Koiti Hasida of Sony Computer Science Laboratories on the construction of a dialog-based information retrieval system.\\

\noindent\textbf{Multi-Agent Systems Laboratory }\hfill Research Assistant\\
Amherst, MA\hfill  September 2000–June 2001\\\\
I collaborated with Professors Victor Lesser and Beverly Woolf on a project which leveraged information retrieval techniques in a distributed educational system. We developed a system to infer symbolic and planning primitives from statistical text analysis.\\

\noindent\textbf{Bell \& Howell Information and Learning }\hfill Research Assistant\\
Ann Arbor, MI\hfill  May 1999–August 2000\\\\
I worked on the implementation of information retrieval research in a real world search system. Tasks included system design and engineering and statistical evaluation of system robustness.\\


\section{Teaching \\Experience}\noindent\textbf{Web Search Engines} \\
New York University (Courant)\hfill Spring 2013, Fall 2014, Fall 2016\\\\
Co-instructed a graduate level course on web search engines.  Classes consisted of lectures with evaluation based on homeworks, exams, and a final project. \\

\noindent\textbf{Web Search} \\
Asian Summer School in Information Access (ASSIA 2013)\hfill Summer 2013\\\\
Graduate level lecture on web search engines.  \\


\noindent\textbf{Experimental Design for Information Systems} \\
University of Trento\hfill Summer 2012\\\\
Co-instructed a graduate level course on information retrieval and data mining.  Classes included a combination of lectures and paper discussion. \\

\noindent\textbf{Advanced Information Retrieval and Databases} \\
New York University (Polytechnic)\hfill Spring 2011\\\\
Co-instructed a graduate level seminar on advanced information retrieval.  Classes included a combination of lectures and paper discussion. \\

\noindent\textbf{Information Retrieval} \\
University of Massachusetts Amherst\hfill Fall 2006\\\\
Prepared slides and provided lectures for a graduate-level information retrieval course. Instructor: Professor James Allan.\\

\noindent\textbf{Applied Information Theory} \\
University of Massachusetts Amherst\hfill Fall 2006\\\\
Provided grading support for a graduate-level information theory course. Instructor: Professor Erik Learned-Miller.\\

\noindent\textbf{Databases} \\
University of Massachusetts Amherst\hfill Spring 2006\\\\
Provided teaching assistance for a graduate-level database course. This included grading home works, holding office hours, and maintaining a course web page. Instructors: Professors Yanlei Diao and Gerome Miklau.

\section{Supervision\\Experience}\noindent\textbf{Former Interns}\\
Jaime Arguello (2009, 2010), Associate Professor, University of North Carolina.\\
Ahmed Hassan (2009), Research Manager, Microsoft Research.\\
Jangwon Seo (2010), Software Engineer, Google.\\
Qi Guo (2011), Software Engineer, Google.\\
Matthew Ekstrand-Abeug (2013), Software Engineer, Google.\\
Pavel Metrikov (2013), Data Scientist, Microsoft.\\
Teresa Bracamonte (2013), Software Engineer, Equifax.\\
Rishabh Mehrotra (2016),  Research Scientist, Spotify.\\

\vspace{\baselineskip}

\noindent Ioannis Paparrizos (2014), PhD Student, Columbia University.\\
Chris Kedzie (2015), PhD Student,  Columbia University.\\
David Abel (2015), PhD Student, Brown University.\\
Cristina Garbacea (2016), PhD Student, University of Michigan.\\
Jesse Anderton (2017), PhD Student,  Northeastern University.
	
\vspace{\baselineskip}
\noindent\textbf{PhD Examiner}\\
Jaime Arguello, Carnegie Mellon University, 2011.\\
Maria-Hendrike Peetz, University of Amsterdam, 2015.\\
Matthew Ekstrand-Abeug, Northeastern University, 2017.\\
Jesse Anderton, Northeastern University, \emph{in progress}.\\
Rodrigo Nogueira, New York University, \emph{in progress}.

\setlength{\leftmargini}{0em}
 \section{Publications}\noindent\textbf{Metrics}\\
\noindent\begin{center}\begin{tabular}{lccc}
&\href{http://dl.acm.org/author\_page.cfm?id=81551748956}{\textit{ACM Digital Library}}&\href{https://www.scopus.com/authid/detail.uri?authorId=55605195900}{\textit{Scopus}}&\href{http://scholar.google.com/citations?user=212SLn0AAAAJ}{\textit{Google Scholar}}\\
\hline
articles&58&70&89\\
citations&1059&1667&3691\\
citations/article&18.26&27.11&41.47\\
h-index&17&24&31
\end{tabular}
\end{center}

\vspace{\baselineskip}
\noindent\textbf{Thesis}\\
\begin{verse}
 \bibentry{diaz:thesis}
\end{verse}
\vspace{\baselineskip}
\noindent\textbf{Chapter}\\
\begin{verse}
 \bibentry{diaz:vertical-chapter}
\end{verse}
\vspace{\baselineskip}
\noindent\textbf{Journal}\\
\begin{verse}
\bibentry{SWIRL2018}
\end{verse}
\begin{verse}
\bibentry{white:result-prefetching-tois}
\end{verse}
\begin{verse}
\bibentry{diaz:worst-practices}
\end{verse}
\begin{verse}
\bibentry{rothschild:twitter-panel}
\end{verse}
\begin{verse}
\bibentry{imran:survey2015}
\end{verse}
\begin{verse}
\bibentry{diaz:forum2014}
\end{verse}
\begin{verse}
\bibentry{goldstein:ad-distract-journal}
\end{verse}
\begin{verse}
\bibentry{hemant:donations}
\end{verse}
\begin{verse}
\bibentry{tois:elad}
\end{verse}
\begin{verse}
\bibentry{tist:twitter}
\end{verse}
\begin{verse}
\bibentry{Diaz:Regularizing-query-based-retrieval}
\end{verse}
\begin{verse}
\bibentry{fdiaz:temporal-journal}
\end{verse}

% \newpage

\vspace{\baselineskip}
\noindent\textbf{Preprints}\\
\begin{verse}
\bibentry{olteanu:survey}
\end{verse}

\vspace{\baselineskip}
\noindent\textbf{Conference}\\
\begin{verse}
\bibentry{mehrotra:fair-marketplace}
\end{verse}
\begin{verse}
\bibentry{garcia-gathright:sigir2018}
\end{verse}
\begin{verse}
\bibentry{mehrotra:demographics-of-search}
\end{verse}
\begin{verse}
\bibentry{alonso:timelines-websci}
\end{verse}
\begin{verse}
\bibentry{mitra:ndrm}
\end{verse}
\begin{verse}
\bibentry{ekstrand:ts-eval}
\end{verse}
\begin{verse}
\bibentry{goel:twitter-language-change}
\end{verse}
\begin{verse}
\bibentry{diaz:altr}
\end{verse}
\begin{verse}
\bibentry{diaz:local-w2v}
\end{verse}
\begin{verse}
\bibentry{diaz:sigir2016}
\end{verse}
\begin{verse}
\bibentry{kedzie:ijcai2016}
\end{verse}
\begin{verse}
\bibentry{diaz:pqr}
\end{verse}
\begin{verse}
\bibentry{arguello:ecir2016}
\end{verse}
\begin{verse}
\bibentry{diaz:crm}
\end{verse}
\begin{verse}
\bibentry{kedzie:acl2015}	
\end{verse}
\begin{verse}
\bibentry{metrikov:image-ads}
\end{verse}
\begin{verse}
\bibentry{icwsm2014:crisislex}
\end{verse}
\begin{verse}
\bibentry{Shokouhi:sigir2014}
\end{verse}
\begin{verse}
\bibentry{golbus:www2014}
\end{verse}
\begin{verse}
\bibentry{fdiaz:robust-mouse-tracking-models}
\end{verse}
\begin{verse}
\bibentry{imran:iscram2013}
\end{verse}
\begin{verse}
\bibentry{qi:temporal-summarization}
\end{verse}
\begin{verse}
\bibentry{arguello:fedltr}
\end{verse}
\begin{verse}
\bibentry{yomtov:detachment}
\end{verse}
\begin{verse}
\bibentry{yomtov:location}
\end{verse}
\begin{verse}
\bibentry{seo:quicklinks}
\end{verse}
\begin{verse}
\bibentry{arguello:fedeval}
\end{verse}
\begin{verse}
\bibentry{bai:mlr-xfer}
\end{verse}
\begin{verse}
\bibentry{diaz:dating}
\end{verse}
\begin{verse}
\bibentry{arguello:vertical-xfer}
\end{verse}
\begin{verse}
\bibentry{dong:twitter-mlr}
\end{verse}
\begin{verse}
\bibentry{dong:y-recency}
\end{verse}
\begin{verse}
\bibentry{arguello:classification-dir}
\end{verse}
\begin{verse}
\bibentry{hassan:geo-newsdd}
\end{verse}
\begin{verse}
\bibentry{arguello:vertical-selection}
\end{verse}
\begin{verse}
\bibentry{diaz:online-vertical-selection}
\end{verse}
\begin{verse}
\bibentry{diaz:online-newsdd}
\end{verse}
\begin{verse}
\bibentry{diaz:rf-regularization}
\end{verse}
\begin{verse}
\bibentry{diaz:xling-regularization}
\end{verse}
\begin{verse}
\bibentry{diaz:lsr-bounds}
\end{verse}
\begin{verse}
\bibentry{diaz:autocorrelation}
\end{verse}
\begin{verse}
\bibentry{diaz:pseudoaligned-corpora}
\end{verse}
\begin{verse}
\bibentry{diaz:ee}
\end{verse}
\begin{verse}
\bibentry{diaz:regularization}
\end{verse}
\begin{verse}
\bibentry{diaz:sigir04}
\end{verse}
\begin{verse}
\bibentry{kelly:users}
\end{verse}
\begin{verse}
\bibentry{diaz:wearables}
\end{verse}
%\bibentry{match-making-retrieval}
\vspace{\baselineskip}
\noindent\textbf{Workshop}\\
\begin{verse}
\bibentry{bird:ethics-of-autonomous-experimentation}
\end{verse}
\begin{verse}
\bibentry{abel:geql}
\end{verse}
\begin{verse}
\bibentry{diaz:kdd2014}
\end{verse}
\begin{verse}
\bibentry{imran:www2013}
\end{verse}
\begin{verse}
\bibentry{louis:newsdd-similarity}
\end{verse}
\begin{verse}
\bibentry{hassan:geo-mlr}
\end{verse}
\begin{verse}
\bibentry{umass:robust2005}
\end{verse}
\begin{verse}
\bibentry{umass:hard2005}
\end{verse}
\begin{verse}
\bibentry{umass-hard2004}
\end{verse}
\vspace{\baselineskip}
\noindent\textbf{Tutorial}\\
\begin{verse}
  \bibentry{monitor:recsys-tutorial}
\end{verse}
\begin{verse}
  \bibentry{crisis-informatics-tutorial}
\end{verse}
\begin{verse}
  \bibentry{temporal-dynamics-tutorial}
\end{verse}
\begin{verse}
  \bibentry{fedsci-tutorial-2}
\end{verse}
\begin{verse}
  \bibentry{fedsci-tutorial}
\end{verse}

\section{Patents}\noindent\textbf{Granted}\\
\begin{verse}
\bibentry{diaz2012system:1}
\end{verse}
\begin{verse}
\bibentry{diaz:audience-response}
\end{verse}
\begin{verse}
\bibentry{chang2013cross}
\end{verse}
\begin{verse}
\bibentry{jones2013system}
\end{verse}
\begin{verse}
\bibentry{diaz2012system:2}
\end{verse}
\begin{verse}
\bibentry{jones2009system}
\end{verse}

\vspace{\baselineskip}
\noindent\textbf{Applied}\\
\begin{verse}
 \bibentry{josifovski2013method}


 \bibentry{yom2012using}

\end{verse}
\vspace{1em}
% \section{Recognition}\noindent\vspace{-2\baselineskip}\subsection{Awards}
\section{Recognition}\noindent\textbf{Awards}\\
British Computer Society Karen Sp{\"a}rck Jones Award, 2017.  \\
Finalist for Paul E. Green Award, Journal of Marketing Research, 2015.  \\
Best Paper Award, ISCRAM 2013.  \\
Best Paper Nomination, SIGIR 2011.  \\
Best Student Paper Award, ECIR 2011.\\
Best Paper Award, SIGIR 2009.  \\
Best Paper Award, WSDM 2009.

% \subsection{Fellowships}
\vspace{\baselineskip}
\noindent\textbf{Fellowships}\\
Institute for Pure and Mathematics, UCLA, Postdoctoral Fellow, 2006.\\
University of Massachusetts Opportunity Fellowship, 2000–2001.\\
National Science Foundation Summer Institute in Japan, 2001.

% \subsection{Invited Meetings and Presentations}
\vspace{\baselineskip}
\noindent\textbf{Invited Meetings and Presentations}\\
New York University Text as Data Seminar, 2017.\\
Columbia University Data Science Industry Summit, 2015.\\
Asian Summer School in Information Access, 2013.\\
NSF Task-Based Search Workshop, 2013.\\
Strategic Workshop on Information Retrieval in Lorne, 2012.\\
International Conference on Machine Learning, Invited Cross-Conference Session, 2011\\
Networks and Network Analysis for the Humanities: An NEH Institute for Advanced Topics in Digital Humanities, 2010.

\section{Service}\noindent\textbf{Organization}\\
SIGIR, General Co-Chair, 2021.\\
FAT*, Program Co-Chair, 2019.\\
WSDM, General Co-Chair, 2014.\\
TREC Web Track, Co-Organizer, 2013-2014.\\
TREC Temporal Summarization Track, Co-Organizer, 2013-2014.\\
NTCIR Recipe Search Track, Co-Organizer, 2014.\\
SIGIR Workshop on Time-aware Information Access, Co-Organizer, 2012-2014.\\
SIGIR Workshop on Social Web Search and Mining: Analysis of User Generated Content Under Crisis, Co-Organizer, 2011.
	
\vspace{\baselineskip}
\noindent\textbf{Reviewing}\\
SIGIR Awards Chair, Editor, 2014-2016\\
SIGIR Forum, Editor, 2012-2014\\
SIGIR PC 2008-2010\\
SIGIR SPC 2011-present\\
CIKM Vice Chair 2009\\
CIKM PC 2010-2011, 2014\\
CIKM SPC 2012\\
WSDM PC 2010-2011\\
WSDM SPC 2012-present\\
WWW PC 2011-2014\\
WWW Track Chair 2017\\
ICML 2010-2012\\
EMNLP PC 2010\\
WISE 2010






%\\\\
\section{Skills}\noindent Extensive programming C/C++ experience on Unix platforms. Extensive experience with  map/reduce computing (Hadoop).  Experience with GNU Scientific Library, Matlab/Octave, R, Python, Perl, Java, and scientific computing in a clustered environment. English and Spanish fluency.
\bibliographystyle{plain}


\section{References}\noindent \emph{Available on request.}

\nobibliography{fdiaz} 

\end{document}



